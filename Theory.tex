\RequirePackage{ASetup}

\begin{document}
\begin{center}
\textbf{\Large{Computing Space and Time Autocorrelated Standard Errors}}\\
%\label{sect:star_errors}
Jordan Adamson, \url{https://github.com/Jadamso/STrollR}
\end{center}
The standard errors for cellular data over time can be corrected for autocorrelation over space and time. Denote $\Sigma_{i}$ as the autocorrelation over time within a cell, and $\Sigma_{t}$ as the autocorrelation over space within a time period. Then the covariance matrix for the entire set can be written as separable,
%\begin{eqnarray*}
$\Sigma_{} = \Sigma_{I} + \Sigma_{T}$
%\end{eqnarray*}
, where $\Sigma_{I}$ denotes $\Sigma_{i}$ for all individuals and $\Sigma_{T}$ denotes $\Sigma_{t}$ for all time periods.\footnote{Note that either $\Sigma_{i}$ or $\Sigma_{t}$ are specified with a 0 diagonal as to not double count the variance for cell $g$ in both matrices.} To estimate $\Sigma_{t}$,
% two methods are used in practice: estimating the spatial autocorrelation process \parencite{KP1999} or correcting the standard errors post-estimation \parencite{Conley1999}.
one can use a weight $k_{i,j}$ between observations $i$ and $j$ based on their distance. For example, the Bartlett kernel, also known as the tent or triangle kernel, can create very sparse covariance matrices, as $k_{i,j}=0$ for many $i,j$. To exploit this property, I first formulate the `meat' of the variance estimator,
\begin{eqnarray*}
X' K \circ \hat{e} \hat{e}' X
= X' K \circ E \circ E'X 
= X' K \circ (\mathbf{1} * \hat{e}) \circ ( \mathbf{1} * \hat{e})' X
= X' (\hat{e} * (K * \hat{e})' )' X
\end{eqnarray*}

,where $'$ denotes transpose, $K$ denotes the sparse weights matrix, $\circ$ denotes the Hadamard product of matrices, $*$ denotes the column-wise Hadamard product of a matrix and a vector, and $\mathbf{1}$ denotes a matrix of ones. Specifically, the kernel-weighted residuals, $K \circ \hat{e} \hat{e}'$, are 
\begin{eqnarray*}
\hspace*{-1em}
\left[\begin{array}{cccc}
k_{11}\hat{e}_{1}\hat{e}_{1} & k_{12}\hat{e}_{1}\hat{e}_{2} & \hdots & k_{1N}\hat{e}_{1}\hat{e}_{N}  \\
k_{12}\hat{e}_{1}\hat{e}_{2} & k_{22}\hat{e}_{2}\hat{e}_{2} &	     & 		     					\\
\vdots						 &   				  			& \ddots &  	        					\\
k_{1N}\hat{e}_{1}\hat{e}_{N} & k_{2N}\hat{e}_{2}\hat{e}_{N} & \hdots & k_{NN}\hat{e}_{N}\hat{e}_{N}
\end{array}\right] 
=
\left[\begin{array}{cccc}
k_{11} & k_{12} & \hdots  & k_{1N}  \\
k_{12} & k_{22} & 	      & 		\\
\vdots	&   	& \ddots  &	 	\\
k_{1N} & k_{2N} & \hdots  & k_{NN}
\end{array}\right]
\circ
\left[\begin{array}{cccc}
\hat{e}_{1} & \hat{e}_{1} & \hdots  & \hat{e}_{1}  \\
\hat{e}_{2} & \hat{e}_{2} & 	    & 		       \\
\vdots		&   		  & \ddots  &  	           \\
\hat{e}_{N} & \hat{e}_{N} & \hdots  & \hat{e}_{N}
\end{array}\right]
\circ
\left[\begin{array}{cccc}
\hat{e}_{1} & \hat{e}_{2} & \hdots  & \hat{e}_{N} \\
\hat{e}_{1} & \hat{e}_{2} & 	    &			  \\
\vdots      &			  & \ddots  &			  \\
\hat{e}_{1} & \hat{e}_{2} & \hdots  & \hat{e}_{N}
\end{array}\right]
\end{eqnarray*}


By reformulating the algebra in terms of Hadamard products of sparse matrices, thousands or millions of operations are computed, rather than billions or trillions. This is because once $K$ is separated from the residuals, it only needs to be computed once for all time periods, multiplying the outer product of the residuals by using techniques that exploit its sparseness. While I use this procedure for space-time correlations, this procedure is also practical in other applications. For example, letting K be a binary matrix indicating clusters.
\end{document}

% http://www.pnas.org/content/107/35/15367.full
% http://www.trfetzer.com/using-r-to-estimate-spatial-hac-errors-per-conley/
% Kelejian and Prucha, 1999 # https://www.jstor.org/stable/pdf/2648817.pdf

% Conley,1999 # http://www.sciencedirect.com.libproxy.clemson.edu/science/article/pii/S0304407698000840
% Journal of Econometrics 140 (2007) 131–154, HAC estimation in a spatial framework, Harry H. Kelejian, Ingmar R. Prucha http://econweb.umd.edu/~prucha/Papers/JE140(2007b).pdf
% Specification and estimation of spatial autoregressive models with autoregressive and heteroskedastic disturbances

